\chapter{Kinematik}







\section{Eindimensionale Bewegung}
\label{kap_kinematik_eindimensional}
\index{eindimensionale Bewegung}

Also Bewegung l"angs einer geraden Linie.

Man unterscheidet zwischen der \textbf{Gleichf"ormigen
  Bewegung}\index{gleichf"ormige Bewegung} mit konstanter
Geschwindigkeit ($v = \frac{\Delta s}{\Delta t} = \frac{\diff x}{\diff
  t}= const$) und \textbf{nicht-gleichf"ormiger Bewegung}, bei der die
Geschwindigkeit eben \emph{nicht} konstant ist. Hier kann man die
mittlere Geschwindigkeit,
bzw. \textbf{Momentangeschwindigkeit}\index{Momentangeschwindigkeit}
\begin{equation}
     v = \frac{\diff s}{\diff t}
\end{equation}
bestimmen.\footnote{Man kann sagen, dass die Momentangeschwindigkeit
  auch eine gemittelte Geschwindigkeit ist, nur dass man das
  Zeitintervall $\Delta t$, "uber welchem man mittelt, sehr klein
  macht: $\Delta t \to 0$.} Bei dieser Bewegung handelt es sich um
eine \textbf{beschleunigte Bewegung}, wobei man die Beschleunigung
bestimmen kann:
\begin{equation}
\boxed{
     a = \frac{\diff v}{\diff t} = \frac{\diff^2 s}{\diff t^2}
}
\end{equation}



Um entsprechend aus einer gegebenen Beschleunigung $a(t)$ die
Momentangeschwindigkeit und die bisher zur"uckgelegte Strecke zu
berechnen, integriert man die Beschleunigung "uber die Zeit:
\begin{equation}
   v(t) = \int_0^t a(\hat t) \diff \hat t \text{ und } s(t) = 
   \int_0^t v(\hat t) \diff \hat t
\end{equation}
\begin{Wichtig}
   Wichtig ist dabei, dass man beim Integrieren auch sch"on brav die
   Integrationskonstanten mitnimmt -- sie stellen
   Anfangsgeschwindigkeit $v_0$ bzw. bereits zur"uckgelegte Strecke
   $s_0$ dar!
\end{Wichtig}







\section{R"aumliche Bewegung}
\index{R"aumliche Bewegung}


\begin{Def}[Trajektorie]\index{Trajektorie}
   Ist die Kurve im Raum, auf der sich ein (Massen)Punkt bewegt. Man
   bezeichnet sie oft mit $\gamma$ oder $\vec r(t)$
\end{Def}
Dabei ist $\gamma$ eine Funktion (oder
\emph{\index{Parametrisierung}Parametrisierung}) mit $\gamma : \mathbb
R \to \mathbb R^3: ~ t \mapsto \Ve r(t)$ -- und so wird $\vec r = \vec
r(t)$ auch Trajektorie genannt.

Bei Bewegungen im Raum kann man im Allgemeinen die Definitionen von
Kap. \ref{kap_kinematik_eindimensional} verwenden. Hier verwendet man
eben statt der einfachen Strecke $s$ die Raumkurve $\Ve r$.

\begin{Wichtig}[Ableiten von Vektoren]
   \index{Vektor!ableiten}\index{Ableiten von Vektoren} Man leitet
   einen Vektor ab, indem man die Komponenten ableitet:
     $$
      \Ve s = \begin{pmatrix}
                   x\\y\\z
              \end{pmatrix} ~ \Rightarrow ~
              \dot{\Ve{s}} = \begin{pmatrix}
                                  \dot x\\\dot y\\\dot z
                             \end{pmatrix}
     $$
\end{Wichtig}


Nun ist im noch zu beachten, dass die (r"aumliche) Geschwindigkeit $\Ve
v$ stets \emph{in Richtung der Bahntangente} zeigt, w"ahrend die
(r"aumliche) Beschleunigung $\Ve a$ im allgemeinen \emph{nicht in
  Richtung der Bahntangente} zeigt. D.h. $\Ve a$ kann sowohl
Komponenten in Richtung der Bahntangente haben (diese nennen wir $\Ve
a_T$), als auch senkrecht dazu (diese entsprechend $\Ve a_N$ -- das
"`$N$"' steht f"ur \emph{normal}).

Mathematisch kann man das herleiten, indem man die Geschwindigkeit $\Ve
v$ beschreibt als Skalar $v$ und Einheitsvektor $\Ve u_T$, der zur
Bahnkurve des Teilchens jeweils tangential steht:
$$
\Ve v = v \cdot \Ve u_T ~ \Rightarrow ~ \dot{\Ve{v}} = \Ve a =
\frac{\diff}{\diff t} (v \cdot \Ve u_T) = \underbrace{\frac{\diff
    v}{\diff t} \cdot \Ve u_T}_{\Ve a_T} + \underbrace{v \cdot
  \frac{\diff \Ve u_T}{\diff t}}_{\Ve a_N}
$$

Nun muss man untersuchen, wie $\frac{\diff \Ve u_T}{\diff t}$ aussieht. 
Dazu unterscheidet man zwei Spezialf"alle:

\begin{description}[\setlabelstyle{\bfseries\slshape}]
\item[Geradlinige Bahn] dann ist $\frac{\diff \Ve u_T}{\diff t} = 0$
   da $\Ve u_T = const$. Dann haben $\Ve v$ und $\Ve a$ die selbe
   Richtung -- und man ist wieder beim eindimensionalen Fall wie in
   Kap \ref{kap_kinematik_eindimensional} gelandet.
\item[Gekr"ummte Bahn] Hier n"ahert man die Kr"ummung durch einen Kreis
   an. Hier zeigt die Ableitung der Tangente radial zum
   Kreismittelpunkt -- also senkrecht zur Tantente. So ergibt sich
   also eine Komponente $\Ve a_N \neq \Ve 0$.
\end{description}








\section{Superpositionsprinzip}
\index{Superposition}


\begin{Def}[Superpositionsprinzip]
   Diese Prinzip besagt, dass man eine Bewegung in unabh"angige
   Teilbewegungen zerlegen kann.
\end{Def}

So wird beispielsweise ein \textbf{Schiefer Wurf} zerlegt in eine
(nach unten) beschleunigte Bewegung und eine gleichf"ormige Bewegung in
der Horizontalen.

Das Superpositionsprinzip ist immer (im Sinne von "`genau dann wenn"')
bei \emph{\index{Lineare Theorie}Linearen Theorien} anwendbar.




\section{Kreisbewegung I}
\label{kap_kreisbewegung-i}
\index{Kreisbewegung}

Wir betrachten hier eine gleichf"ormige Kreisbewegung (mit Radius $R$),
d.h. dass die Winkelgeschwindigkeit $\omega = \dot \varphi$ ($\varphi$
ist der im Kreis "uberstrichene Winkel im Bogenma"s) konstant ist. Dabei
wird $\omega$ auch als Vektor interpretiert: Er steht senkrecht auf
der Kreisbewegung und folgt der \textbf{Rechten Faust Regel}: D.h. 
die Finger der rechten Faust zeigen in die Richtung, in die das
Massenteilchen rotiert, dann zeigt der ausgestreckt Daumen in die
Richtung des Vektors $\Ve \omega$.

Man kann die Bahnkurve eines solchen Teilchens parametrisieren durch
$$
 \Ve r = \begin{pmatrix}
              R \cdot \cos \varphi(t)\\
              R \cdot \sin \varphi(t)
         \end{pmatrix}
$$

Das Bogenma"s eines Winkels ist definiert als
\begin{equation*}
   \varphi = \frac{s}{r}
\end{equation*}
wobei $r$ der Radius und $s$ die \emph{Kreisbogenl"ange} eines
Kreisbogens ist, der den "Offnungswinkel $\varphi$ bestitzt. F"ur eine
Rotation um einen konstanten Radius $r$ ist deswegen die
\index{Winkelgeschwindigkeit}\textbf{Winkelgeschwindigkeit}:
\begin{equation}
   \label{eqn_winkelgeschwindigkeit_kreis}
   \omega = \dot \varphi = \frac{\dot s}{r} = \frac{v}{r}
\end{equation}
Wenn wir nun einen \emph{Vektor} $\vec v$ untersuchen, so sehen wir,
dass dieser seine Richtung st"andig "andert -- weil $v = \omega \cdot r$
ist, und $\omega$ und $r$ konstant sind, ist die \emph{L"ange} des
Vektors jedoch konstant. 

Untersucht man den Geschwindigkeitsvektor
eines mit $\omega = \const$ rotierenden Teilchens zu zwei Zeiten, so
ist der Winkel $\diff \varphi$ nicht nur der "Offnungswinkel zwischen den
beiden Ansatzpunkten der Vektoren, sondern auch der Winkel zwischen
den Vektoren. "Uberlegt man weiter, dass die Vektorspitze demnach einen
Kreisbogen der L"ange
\begin{equation*}
   \| \diff \vec v \| = v \cdot \diff \varphi
\end{equation*}
zur"uckgelegt hat, so findet man, in dem man
\eqref{eqn_winkelgeschwindigkeit_kreis} nach $v$ aufl"ost, einsetzt und
durch $\diff t$ teilt:
\begin{equation}
   \label{eqn_winkelbeschleunigung_kreis}
\left \|   \frac{\diff }{\diff t} \vec v \right \| =  \dot \varphi r
\cdot \frac{\diff \varphi}{\diff t} = {\dot \varphi}^2 r = \omega^2
\cdot r = a_Z
\end{equation}
Man erh"alt also die
\textbf{\index{Zentripetalbeschleunigung}Zentripetalbeschleunigung}.
Multipliziert man diese nun mit $m$ erh"alt man wegen $F = ma$ die
Zentripetalkraft zu
\begin{equation}
   \label{eqn_zentripetalkraft_kreis}
   F_z = m \cdot \omega^2 \cdot r
\end{equation}



Bei der Kreisbewegung ergeben sich nun folgende Gr"o"sen:
\begin{description}[\setlabelstyle{\bfseries\slshape}]
     \item[Periodendauer] $T = \frac{2\pi}{\omega}$
     \item[Frequenz] $f = \nu = \frac{1}{T} = \frac{\omega}{2\pi}$
     \item[Bahngeschwindigkeit] $\boxed{v = \frac{2\pi R}{T} = \omega \cdot R}$
bzw. $\vec v =  \vec \omega \times \vec r$
     \item[Zentripetalbeschleunigung] $\Ve a_Z = \frac{v^2}{R} \cdot
        \Ve u_N ~\Rightarrow ~ \boxed{\|\Ve a_Z\| = \omega^2 \cdot R}$
        \\gilt nur f"ur $\omega = const$!
 \end{description}


\begin{Wichtig}[$\nu \neq v$]
   Der griechische Buchstabe $\nu$ ("`n"u"') gibt meistens Frequenzen
   an, w"arend das lateinische $v$ ("`vau"') oft f"ur Geschwindigkeiten
   verwendet wird. Nicht verwechseln!
\end{Wichtig}
















