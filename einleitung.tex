\chapter{Einleitung: Erkenntnisprozess}
\index{Erkenntnisprozess}

In den Naturwissenschaften verl"auft der Erkenntnisprozess im
Allgemeinen "uber die folgenden Stufen:

\begin{description}[\setlabelstyle{\bfseries\slshape}]
\item[Experiment] Von Beobachtungen werden Gesetzm"a"sigkeiten
   abgeleitet
\item[Induktion] Von den Spezellen Gesetzm"a"sigkeiten schlie"st man auf
   allgemeinere Zusammenh"ange
\item[Formulierung] Die allgemeinen Zusammenh"ange werden in einer
   Formel oder als Gesetz \emph{formuliert}
\item[Deduktion] Von Allgemeinen Fall wird auf einen speziellen
   geschlossen und dieser Spezielle durch ein weiteres Experiment
   \emph{verifiziert} oder \emph{falsifiziert}.
\end{description}

Physik ist so \emph{meistens} vorhersagbar -- es gibt aber auch
Gegenbeispiele, bei denen die \emph{Wahrscheinlichkeit} eine wichtige
Rolle spielt (bspw. Quantenphysik).















