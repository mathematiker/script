\chapter{Vorwort vom Editor}

Ich habe eigentlich nicht viel zu sagen.  Ich möchte mich eigentlich
bloß herzlich bei Michael Kopp bedanken, dass ihr mir sein Skript zur
Verfügung gestellt hat, um es ein wenig an die Inhalte von diesem Jahr
anzupassen. Besucht doch auch mal seine Webseite:
\url{http://www.extycion.de/physik}.  Da gibt es echt tolle Dinge.

Würde mich über Korrekturen freuen,  freundet euch am besten mit
\texttt{git} an. Über die Plattform könnt ihr Änderungen direkt
eintragen,  bzw. über \emph{push}-Request die Korrekturen automatisch
miteinbinden lassen.  Wenn ihr euch aktiv beteiligen wollt,  meldet
euch bitte bei mir: \href{mailto:math3matik3r@hotmail.de}{math3matik3r@hotmail.de}.
Ich erkläre gerne Einzelheiten.

Ich möchte betonen, dass dies kein offizielles Skript ist.  Und naja,
Fehler kann ich halt auch nicht ausschließen, zumal ich das Ding ja
gar nicht geschrieben habe. Aktuelle Versionen vom Skript könnt ihr
auf \url{https://github.com/mathematiker/script} ziehen und
kompilieren.  Beachtet die \emph{README}-File, ihr müsst \fbox{LATEX $
\to $ DVI ($ \to $ PS) $ \to $  PDF} kompilieren.  Die \emph{Makefile}
könnt ihr mit \emph{GNU Make} ausführen (in der Regel in
Linux-Distributionen vorinstalliert).
