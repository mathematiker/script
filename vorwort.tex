\chapter{Vorwort vom Autor}
\label{kap_vorwort}

Dieses Script war ein ziemlicher Haufen Arbeit! Und ich habe sie mal
mehr und mal weniger gerne gemacht -- aber ich \emph{habe} sie
gemacht.
Es ist der Versuch, aus meinen Vorlesungsmitschrieben einen
strukturierten, "ubersichtlichen, mathemtisch korrekt(er)en Text zu
gestalten, mit dem man anst"andig Lernen kann...

Ich habe mir dabei M"uhe gegeben, aber auch wenn ich in B"uchern und
der allwissenden \textsc{Wikipedia} verglichen habe, so gab es doch
manchmal verschiedene Ansichten "uber das selbe Problem. Ich habe mir
dann meine eigene, f"ur mich logische Erkl"arung aus verschiedenen
Quellen zusammengebastelt. Deshalb habe ich auch manches, was wir in
der Vorlesung nicht hatten, was mich aber intertessierte, aufgenommen
(die Kapitel mit dem $(\bigstar)$ kann man problemlos "uberspringen).


Anyways, ich w"unsche allen Benutzern viel Freude damit, dass es auch
n"utzlich ist. Und ich weise ebenso galant die Verantwortung zur"uck,
sollte ich Fehler gemacht haben: Irren ist m"annlich; auch wenn ich
nach bestem Wissen und Gewissen gearbeitet habe.

Ach ja: Die (zahlreichen) Rechtschreib-, Grammatik- und Tippfehler
sollen den Text humoristisch aufwerten ("`Studiert und kann nicht mal
Deutsch"') und sind voll beabsichtigter,  wichtiger Bestandteil des
Werks!

\abs
Noch ein paar technische Details: Kurven hei"sen bei mir in
Kurvenintegralen meistens $\gamma$ (oder selten $C$). Manchmal findet
man auch $\gamma = \partial A$, dabei bedeutet $\partial$ einfach
\emph{Rand} der Fl"ache $A$. Vektoren haben einen Pfeil dr"uber und
ein Vektor ohne Pfeil ist der Betrag des Vektors (oder ich habe den
Pfeil vergessen) -- bei Skalarprodukten wurden die Pfeile teilweise
weggelassen: ${\vec v}^2 = v^2$. Beispiele, Definitionen und wichtiges
sind in anderer Schriftart bzw. mit grauem Hintergrund hervorgehoben,
besonders interessante Formeln extra eingerahmt. Eigennamen wurden
(meist) mit \textsc{Kapit"alchen}
gesetzt. Vektordifferenzialoperatoren wurden meist einfach mit $\vec \nabla
\cdot$ (Divergenz / Gradient) oder $\vec \nabla \times$ (Rotation)
geschrieben.

\abs {\sl Und zu guter Letzt m"ochte ich mich bei all denen Bedanken,
  die mich auf verschieden dumme Fehler aufmerksam gemacht haben und
  so geholfen haben, die Qualit"at ein wenig zu sichern} (besonders
Michi Bauer, der das Script sogar in die Ferien mitnam, Dominique f"ur
die Beseitigung zahlreicher sprechlicher Verwirrungen und Halbs"atze
und Michi Springer, Johannes und Paul f"ur konsequentes Mitdenken).

\vfill
\hfill \textit{Michael Kopp}

\hfill Besigheim, früher % \today ist unpassend


















