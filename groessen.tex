\chapter{Physikalische Gr"o"sen}

\begin{Def}[Physikalische Gr"o"se]\index{Gr"o"se}\index{Physikalische Gr"o"se}
   besteht stets aus
     $$
     <\text{Zahlenwert}> \cdot <\text{Einheit}>
     $$
\end{Def}
Man versucht sich dabei auf m"oglichst wenige Einheiten -- die
sog. \textbf{Basiseinheiten} -- zu beschr"anken.
\begin{Def}[Basiseinheit]\index{Basiseinheit}
   Grundlage f"ur Physkalische Gr"o"sen und Messwerte; m"ussen "uberall
   bestimmbar bzw. nachpr"ufbar sein und orientieren sich deshalb oft
   an Naturkonstanten.
\end{Def}







\section{Basiseinheiten}



\subsection{Vorsilben}

Zehnerpotenzen modulo 3 werden bei Physikalischen Gr"o"sen durch
\emph{Vorsilben} vor den Einheiten ersetzt:
\begin{center}
% use packages: array
\begin{tabular}{c l c}
\toprule
 Zehnerpotenz & Vorsilbe & Abk"urzung \\
 \midrule
-18 & atto & a\\ 
-15 & fempto & f \\ 
-12 & pico & p  \\
-9  & nano & n  \\
-6  & micro & $\mu$  \\
-3 & milli & m  \\
0  & -- & --  \\
3  & Kilo& K  \\
6  & Mega & M  \\
9  & Giga & G  \\
12 & Terra & T \\
\bottomrule
\end{tabular}
\end{center}







\subsection{Zeit}

\begin{Def}[Sekunde \textbf s]\index{Sekunde}\label{def_sekunde}
   $1s = 9 \, 192\, 631\, 770 \cdot T_A$, wobei $T_A$ die
   Periodendauer des Hyperfein"ubergangs von $^{133}Cs$ ist -- ein sehr
   pr"azise stattfindender Vorgang in C"asiumatomen.
\end{Def}

Um \textbf{sehr kurze} Zetr"aume zu messen, verwendet man schwingende
Quarze oder andere periodische Vorg"ange in Atom(kern)en. F"ur
\textbf{sehr lange} Zeitr"aume dagegen verwendet man bspw. radioaktiv
zerfallende Stoffe und deren Halbwertszeit.





\subsection{L"ange}

\begin{Def}[Meter \textbf m]\index{Meter}\label{def_meter}
   $1m = c_0 \cdot \frac{1}{299\, 792\, 458}\operatorname{s}$ mit der
   (Vakuum)Lichtgeschwindigkeit $c_0$.
\end{Def}

Um \textbf{sehr gro"se} Abst"ande zu messen, benutzt man
Triangulationsverfahren. Bei \textbf{sehr kleinen} Abst"anden
Laserinterferomenter, bei denen man ausnutzt, dass sich Lichtwellen
bei bestimmten Bedingungen ausl"oschen und die Lichtwellenl"ange bekannt
ist.





\subsection{Masse}

\begin{Def}[Kilogramm
   \textbf{Kg}]\index{Kilogramm}\label{def_kilogramm}
   $1Kg = \text{Masse des Urkilogramms}$. Das Urkilogramm ist ein
   Gewicht in Paris, welches das Kilogramm definiert.
\end{Def}

Das Kilogramm ist also eine \emph{willk"urliche} {Definition} und nicht
"uber Natukrkonstanten definiert. Man versucht hier Abhilfe zu
schaffen, indem man bspw. eine monokirstalline Siliciumkugel erstellt,
bei der man genau die Anzahl der Atome bestimmen kann, deren einzelnes
Gewicht bekannt ist -- so k"onnte man eine allgemeine, (unter gro"sem
Aufwand) reproduzierbaer Definition des Kilogramms erstellen.







\subsection{MKS-System}
\label{kap_mks-system}

\begin{Def}[MKS-System]\index{MKS-System}
   Ein \emph{Ma"ssystem}, welches alle in der \emph{Mechanik} wichtigen
   Einheiten auf die drei Grundeinheiten
   \begin{itemize}
   \item Meter
   \item Kilogramm
   \item Sekunde
   \end{itemize}
   zur"uckf"uhrt.
\end{Def}

\index{SI-System} Das \textbf{SI-System} (\emph{System
  Internationale}) stellt dar"uber hinaus weitere Einheiten zur
Verf"ugung, die Grundlage sind f"ur weitere Teilgebiete der Physik:

\begin{description}[\setlabelstyle{\bfseries\slshape}]
\item[Thermodynamik] Absolute Temperatur in Kelvin $\mathbf
   K$\\Stoffmenge in Mol $\mathbf{mol}$
     \item[Elektrodynamik] Stromst"arke in Ampere $\mathbf A$
     \item[Optik] Lichtst"arke in Candela $\mathbf{Cd}$
 \end{description}






\section{Messfehler}\index{Messfehler}

\subsection{Systematischer Messfehler}

\begin{Def}[Systematischer Messfehler]\index{Systematischer
     Messfehler}
     Stets gleiche Abweichung
\end{Def}

\textbf{Ursachen} k"onnen sein:
\begin{itemize}
     \item Fehler im Messinstrument
     \item Rundungen / Vereinfachungen in den Formeln
\end{itemize}




\subsection{Statistische (zuf"allige) Messfehler}

\begin{Def}[Statistische Messfehler]\index{Statistische Messfehler}
   Die Abweichtung ist \emph{zuf"allig} d.h. liefert verschiedene
   Werte.
\end{Def}

\textbf{Ursachen} k"onnen sein:
\begin{itemize}
     \item Ablesefehler durch Beobachter
     \item (statistische) Schwankungen "au"serer Einfl"usse
\end{itemize}

Diese Fehler sind i.A. nicht vermeidbar, jedoch verringerbar. Man
bedient sich dazu statistischer
Auswertungsmethoden\index{Statistik}. Wiederholt man den Versuch oft,
kann man einen \textbf{Mittelwert}
\begin{equation}
     \bar x = \frac{1}{n} \cdot \sum_i x_i
\end{equation}
der $n$ Messwerte $x_i$ bestimmen. 

Weiter gibt die \textbf{Standardabweichung} oder \textbf{Streuung}
\begin{equation}
     \sigma = \sqrt{\frac{1}{n-1} \cdot \sum_i (x_i - \bar x )^2}
\end{equation}
an, wie zuverl"assig $\bar x$ ist -- Messwerte $x_i$ innerhalb der
$\sigma$-Umgebung von $\bar x$ treten mehr als halb so h"aufig auf, wie
$\bar x$ selbst.

Um die W"ahrscheinlichkeit $p$ zu berechnen, mit der ein bestimmter
Messwert $x$ auftreten sollte, verwendet man die
\textsc{Gauss}'sche-Normalverteilung\index{Normalverteilung}:
\begin{equation}
   p(x) = \frac{1}{\sqrt{2\pi} \cdot \sigma} \cdot 
   \exp\left ( \frac{- (x-\bar x)^2}{2\sigma^2} \right )
\end{equation}













